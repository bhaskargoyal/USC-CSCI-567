\documentclass[a4paper, 12pt]{article}

\newcommand{\assignmentAuthor}{Bhaskar Goyal}
\newcommand{\course}{CSCI - 567}
\newcommand{\assignmentName}{HW 02}
\newcommand{\USCID}{6547367383}
\newcommand{\assignmentDate}{July 13, 2021}

\addtolength{\hoffset}{-2.25cm}
\addtolength{\textwidth}{4.5cm}
\addtolength{\voffset}{-2.5cm}
\addtolength{\textheight}{5cm}
\setlength{\parskip}{0pt}
\setlength{\parindent}{15pt}

\usepackage{amsthm}
\usepackage{amsmath}
\usepackage{amssymb}
\usepackage{bm}
\usepackage{tikz}
\usetikzlibrary{automata}
\usepackage{enumitem}
\usepackage{mathabx}

\usepackage{graphicx}
\usepackage{multicol}
\usepackage{ marvosym }
\usepackage{wasysym}
\usepackage{tikz}
\usetikzlibrary{patterns}
\usepackage{fancyhdr}
\usepackage{amssymb,latexsym,amsmath,amsthm}
\usepackage{amsfonts,rawfonts}
\usepackage{thmtools}
\usepackage{systeme}
\usepackage{mathtools}
\usepackage{algorithmic}
\usepackage[colorlinks = true, linkcolor = black, citecolor = black, final]{hyperref}
\usepackage{tikz}
\usetikzlibrary{automata}

\usepackage{graphicx}
\usepackage{multicol}
\usepackage{ marvosym }
\usepackage{wasysym}
\usepackage{tikz}
\usetikzlibrary{patterns}
\usepackage{fancyhdr}
\usepackage{amssymb,latexsym,amsmath,amsthm}
\usepackage{amsfonts,rawfonts}
\usepackage{thmtools}
\usepackage{systeme}
\usepackage{mathtools}

\pagestyle{fancy}
\fancyhf{}
\rhead{\assignmentAuthor \; (USC ID - \USCID)}
\lhead{\course \; \assignmentName}
\rfoot{Page \thepage}
\lfoot{\assignmentDate}

\newcommand{\ds}{\displaystyle}

\setlength{\parindent}{0in}

\declaretheoremstyle[
headfont=\color{blue}\normalfont\bfseries,
notefont=\bfseries, 
notebraces={}{},
% bodyfont=\color{red}\normalfont\itshape,
bodyfont=\normalfont,%\itshape,
%headformat=\NUMBER.~\NAME\NOTE
headformat=\NAME\NOTE
]{colorquestion}

\declaretheorem[
numbered=no,
style=colorquestion,
name=Question
]{question}

\title{\course \; \assignmentName}
\author{\textbf{\assignmentAuthor} \\ \Small{USC ID - \USCID}}
\date{\assignmentDate}

% ----------------------------

% The "stuff" above here is called the preamble of the document.  It sets the margins and loads special packages.  Probably the only reason you would need to edit something above here would be to add a package to do something very specific... but probably everything you need is loaded already

% -----------------------------

\begin{document}
\thispagestyle{plain}
\maketitle
\hrule
\bigskip

% -------------------------
% Question 1
% -------------------------

\begin{proof}[\color{red}{\textbf{Problem 1}: Name}]

\hfill

\begin{enumerate}[label={\color{blue}{\textbf{1.\arabic*})}}]
    \item 
        text \\
        text \\
        text 
        
        
    \item 
        text \\
        text \\
        text 
        
        
    \item 
        text \\
        text \\
        text 
    
\end{enumerate}
\end{proof}



% -------------------------
% Question 2
% -------------------------
\hrule
\bigskip

\begin{proof}[\color{red}{\textbf{Problem 2}: Name}]

\hfill

\begin{enumerate}[label={\color{blue}{\textbf{2.\arabic*})}}]
    \item 
        text \\
        text \\
        text 
        
        
    \item 
        text \\
        text \\
        text 
        
        
    \item 
        text \\
        text \\
        text 
    
\end{enumerate}
\end{proof}
% -------------------------
% Question 3
% -------------------------
\hrule
\bigskip

\begin{proof}[\color{red}{\textbf{Problem 3}: Name}]

\hfill

\begin{enumerate}[label={\color{blue}{\textbf{3.\arabic*})}}]
    \item 
        \textbf{Given:} 
        
    \item 
        text \\
        text \\
        text 
        
        
    \item 
        text \\
        text \\
        text 
    
\end{enumerate}
\end{proof}

% -------------------------
% Question 4
% -------------------------
\hrule
\bigskip

\begin{proof}[\color{red}{\textbf{Problem 4}: Name}]

\hfill

\begin{enumerate}[label={\color{blue}{\textbf{4.\arabic*})}}]
    \item 
        text \\
        text \\
        text 
        
        
    \item 
        text \\
        text \\
        text 
        
        
    \item 
        text \\
        text \\
        text 
    
\end{enumerate}
\end{proof}
\hrule
\bigskip

\end{document}
